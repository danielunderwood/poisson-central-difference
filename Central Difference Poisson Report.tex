\documentclass[10pt,a4paper]{article}
\usepackage[utf8]{inputenc}
\usepackage{amsmath}
\usepackage{amsfonts}
\usepackage{amssymb}
\usepackage{fourier}
\usepackage{cleveref}
\author{Daniel Underwood}
\title{Linear System Solution of Poisson Equation on Dirichlet Boundary}

\newcommand{\crefrangeconjunction}{--}
\begin{document}
\maketitle

\section*{Introduction}
The Dirichlet boundary value problem is a boundary value problem in which the function on the boundary is defined by a function; that is

\begin{subequations}
  \begin{align}
  \hat{\mathcal{D}} u(x_1 , ... , x_n) = f(x_1 , ... , x_n) \,\, \text{for} \,\, (x_1, ..., x_n) \in \Omega & \label{eqn: dirichlet 1}\\
  u(x_1, ..., x_n) = g(x_1, ..., x_n) \,\, \text{for} \,\, (x_1, ..., x_n) \in \partial\Omega \label{eqn: dirichlet 2}&
  \end{align}
\end{subequations}

where $\hat{\mathcal{D}}$ is a linear differential operator consisting of a sum of derivatives of any order including mixed derivatives and $u$, $f$, and $g$ are functions of $n$ variables mapping from $\mathbb{R}^n$ to $\mathbb{R}$. The problem is solved on a domain $\Omega \subset \mathbb{R}^n$ with a boundary $\partial \Omega$. An additional constraint of the Dirichlet problem is that $u$ is a harmonic function. In simpler terms, a Dirichlet problem is a boundary value problem in which the solution value is given in terms of a function, rather than other conditions such as the value of the solution's derivative on the boundary that is given in a Neumann problem.

The Dirichlet problem is very useful in a variety of fields. Differential equations frequently used with Dirichlet boundary conditions include the  Laplace equation, Poisson equation, heat/diffusion, and wave equation. These equations are seen quite often and the latter three are actually all more general forms of the Laplace equation; the Poisson equation is a non-homogeneous form while the heat/diffusion and wave equations add first and second time derivative terms, respectively.

The Laplace equation

\begin{equation}
\Delta u = 0 \label{eqn: laplace}
\end{equation}

is one of the most basic partial differential equations used with Dirichlet boundary conditions. The solutions to this equations are harmonic functions. The Laplace equation has uses in various fields of physics to describe potentials due to forces such as gravity, the electromagnetic force, and fluid forces. It it also the steady-state heat or diffusion equation as well as the steady-state wave equation by taking $u_t = 0$ in \cref{eqn: diffusion} or $u_{tt} = 0$ in the $\Box$ operator of \cref{eqn: wave}.

The Poisson equation

\begin{equation}
\Delta u = f \label{eqn: poisson}
\end{equation}

Is the non-homogeneous form of \cref{eqn: laplace}. The Poisson equation has uses throughout physics as $f$ can be used to represent forces such as gravity, electromagnetism, and fluid forces. In these cases, $u$ represents a potential, which can be related to a force via Newton's second law or can be used in energy analysis. The $f$ term comes from a source. In the case of an electrostatic field, we have $f = - \frac{\rho (r)}{\epsilon_0}$, where $\rho$ is the radial charge density; the gravitational source term is $f = 4 \pi^2 G \rho(r)$, where $G$ is the universal gravitational constant and $\rho$ is the radial mass density.

The heat or diffusion equation

\begin{equation}
u_t - a \Delta u = 0 \label{eqn: diffusion}
\end{equation}

is a frequently used partial differential equation to represent the transfer of heat on the domain of a piece of material or the diffusion of a fluid through another fluid. $a$ is either the heat transfer coefficient or the diffusion coefficient depending on the usage case. It is frequently constant as a simplification, but can vary with position or time in more complicated cases. In these cases, the divergence term affects the transfer function and the equation will take on a slightly different form.

The wave equation

\begin{equation}
\Box u = 0 \label{eqn: wave}
\end{equation}

where $\Box \equiv \frac{1}{c^2} \frac{\partial^2}{\partial t^2} - \Delta$ is the d'Alembert operator with $c$ being the wave speed. The wave equation has usage throughout physics. In classical mechanics, the wave equation is used to represent oscillations and is the start to the large field of wave mechanics. Electromagnetism and quantum mechanics have their own wave equations, being Maxwell's equations for electromagnetism and the Schr{\"o}dinger equation in quantum mechanics.

\section*{Mathematical Theory}


\section*{Computational Details}

\section*{Conclusion}
\end{document}